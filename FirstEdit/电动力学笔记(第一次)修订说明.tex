\documentclass[a4paper,20pt]{article}
\usepackage{geometry}
\usepackage[UTF8]{ctex}
\usepackage{amsmath}
\usepackage{esint}
\usepackage{mathrsfs}
\usepackage{amssymb}

\geometry{left=1.5cm, right=1.5cm, top=1.5cm, bottom=1.5cm}
\setlength{\lineskip}{0.75em}
\setlength{\parskip}{0.75em}


\begin{document}
 \begin{center} 
 \Large \textbf{电动力学笔记(第一次)修订说明}
\end{center}
 
\large 
\quad 自笔者的电动力学笔记原稿开始编辑以来, 已经过去大约1年时间. 笔者有注意到这份笔记在一定程度上受到了肯定, 这给予了笔者维护其的动力. 笔者在此间的学习过程中发现之前的笔记存在诸多不足, 如错别字和理论错误等. 因此修订这份笔记一直在笔者的计划列表中. \par

\quad 随着笔者进学测试的结束, 笔者开始有时间着手对这份笔记进行第一次修订. 当然, 在工作的开始笔者是期待这修订是一劳永逸的, 但无人能不犯错, 对于才疏学浅的笔者而言尤是如此.\par

\quad 这份笔记原稿的完成离不开多方面的支持, 尤其感谢笔者的男友(Farthing)对笔者学习给予了大量精神上的支持, 在笔者学习最困难的时期也给予了笔者最坚实的鼓励. 如果没有这份鼓励, 笔者很大概率无法坚持完成这份笔记, 尤其是在已经做了MindMapping的情况下笔者的惰性更是占据主导地位. 非常感谢openSUSE社区为笔者提供了一如既往稳定的操作系统以完成初稿的编辑. 非常感谢FreeMind的开发者, 他们提供的程序帮助读者完成了电动力学课程第一次系统性的学习, 也使得这份文本笔记的原始导图文件顺利完成. 同样非常感谢Kile和TexLive的开发者们, 如果没有他们的工作, 这份由MindMapping构成的笔记将无法以现在这样较为成熟的文本得以呈现. 另外, Octave和Maxima程序的开发者们辛勤的工作也为笔者的学习提供了方便且不受限制的工具. 感谢Pennsylvania大学提供的备份管理软件, 这使得所有笔记文件得以完整保存至今. 当然, 还要感谢Panasonic和HP公司销售的笔记本计算机, Casio公司销售的科学计算器, Logi公司销售的无线鼠标等. 它们良好的性能让笔者能够随时随地学习和编辑, 不过笔者已经为此付费.\par

\quad 本次修订能够顺利发布, 除了要感谢上述支持笔者完成初稿的个人和组织外, 还要感谢厦门航空飞行员的安全驾驶. \par

\quad 如上述内容重申, 受笔者能力所限, 无法确保本次修订达到预期标准. 若读者有任何建议请参照笔者个人主页的电子邮件地址提出建议. \par

\quad 值得注意并被重申的是, 维护该笔记完全出于笔者业余兴趣并开销笔者业余时间. 笔者接受读者的一切合理建议, 但拒绝一切意见. \par

\quad \par

\quad 敬颂 \\
冬祺\par

\quad\par
\quad\par

\quad Umaru Aya\par
2021年12月30日\par

\end{document}
